\documentclass[useAMS,usenatbib]{mn2e}
\usepackage{hyperref}
\usepackage[utf8]{inputenc}
\usepackage{graphicx,url,times,subfig}
%\usepackage{aas_macros}
\setlength{\paperheight}{297mm}
\setlength{\paperwidth}{210mm}

%To produce a PDF file with hyperlinks, uncomment the following lines
%\usepackage[pdftex,pdfpagemode={UseOutlines},bookmarks,bookmarksopen,
%colorlinks,linkcolor={blue},citecolor={green},urlcolor={red}]{hyperref}

% new if for PDF mode
\newif\ifpdf\ifx\pdfoutput\undefined\pdffalse\else\pdfoutput=1\pdftrue\fi
% set PDF properties if in pdf latex
\ifpdf\hypersetup{
pdftitle={SubHaloes going Notts: The SubHalo-Finder Comparison Project},
pdfauthor={Julian Onions},
pdfkeywords={N-body simulations, haloes evolution, dark matter},
}
\fi

% the finder names
\newcommand{\ahf}{\textsc{ahf}}
\newcommand{\subfind}{\textsc{subfind}}
\newcommand{\asohf}{\textsc{asohf}}
\newcommand{\lanl}{\textsc{lanl}}
\newcommand{\gadget}{\textsc{gadget3}}
\newcommand{\voboz}{\textsc{voboz}}
\newcommand{\mendieta}{\textsc{mendieta}}
\newcommand{\dfof}{\textsc{6dfof}}
\newcommand{\adaptahop}{\textsc{adaptahop}}
\newcommand{\fiestathree}{\textsc{HOT+FiEstAS (3d)}}
\newcommand{\htd}{\textsc{hot3d}}
\newcommand{\hsd}{\textsc{hot6d}}
\newcommand{\fiestasix}{\textsc{HOT+FiEstAS (6d)}}
\newcommand{\rockstar}{\textsc{rockstar}}
\newcommand{\hbt}{\textsc{hbt}}
\newcommand{\hsf}{\textsc{hsf}}
\newcommand{\stf}{\textsc{stf}}

\newcommand{\Fig}[1]{Figure~\ref{#1}}
\def\vmax{$v_{\rm max}$}
\newcommand{\rth}{$R_{200}$}
\newcommand{\subhalos}{subhaloes}

\newlength{\figwidth}
\setlength{\figwidth}{0.43\textwidth}
\newlength{\resplot}
\setlength{\resplot}{0.32\textwidth}

% main paper
\title[SubHalo-Finder Comparison]
{SubHaloes going Notts: The SubHalo-Finder Comparison Project}
\author[Onions et al.]
{Julian~Onions,$^1$\thanks{E-mail: \href{mailto:julian.onions@gmail.com}{julian.onions@gmail.com}}
  Alexander~Knebe,$^2$
  Frazer~R.~Pearce,$^1$ 
  Stuart~I.~Muldrew,$^1$ 
\newauthor
  Hanni~Lux,$^1$
  Steffen~R.~Knollmann,$^2$ 
  Yago Ascasibar,$^2$
  Peter Behroozi,$^{3,4,5}$
\newauthor
  Pascal Elahi,$^{6}$
  Jixian Han,$^{7,6}$
  Michal Maciejewski,$^8$
  M. E. Merch\'{a}n,$^{9}$
\newauthor
  Mark Neyrinck,$^{10}$
  A. N. Ruiz,$^{9}$ 
  M. A. Sgr\'{o},$^{9}$ 
  Volker Springel,$^{11,12}$
 and Dylan Tweed$^{13}$
\\
  $^1$School of Physics \& Astronomy, University of Nottingham, Nottingham, NG7 2RD, UK\\
  $^2$Departamento de F\'{ı}sica Te\'{o}rica, M\'{o}dulo C-15, Facultad de Ciencias, 
  Universidad Aut\'{o}noma de Madrid, 28049 Cantoblanco, Madrid, Spain\\
  $^{3}$Kavli Institute for Particle Astrophysics and Cosmology, Stanford, CA 94309, USA\\
  $^{4}$Physics Department, Stanford University, Stanford, CA 94305, USA\\
  $^{5}$SLAC National Accelerator Laboratory, Menlo Park, CA 94025, USA\\
  $^{6}$Shanghai Astronomical Observatory, 80 Nandan Road, Shanghai, China, 200030 \\
  $^{7}$Institute for Computational Cosmology, Department of Physics, Durham University, South Road, Durham DH1 3LE, UK\\
  $^{8}$Max-Planck-Institut f\"{u}r Astrophysik, Garching,
  Karl-Schwarzschild-Stra\ss e 1, 85741 Garching bei M\"{u}nchen,
  Germany\\
  $^{9}$ Instituto de Astronom\'{ı}a Te\'{o}rica y Experimental (CCT C\'{o}rdoba, CONICET, UNC), Argentina\\
  $^{10}$Department of Physics and Astronomy, Johns Hopkins University, 
  3701 San Martin Drive, Baltimore, MD 21218, USA\\
$^{11}$Heidelberg Institute for Theoretical Studies, Schloss-Wolfsbrunnenweg 35, 69118 Heidelberg, Germany\\ 
$^{12}$Zentrum f\"ur Astronomie der Universit\"at Heidelberg, ARI, M\"{o}nchhofstr. 12-14, 69120 Heidelberg, Germany \\
  $^{13}$Institut d'Astrophysique Spatiale, CNRS/Universite Paris-Sud 11, 91405 Orsay, France \\
}

\begin{document}
%\date{}

\pagerange{\pageref{firstpage}--\pageref{lastpage}} \pubyear{2011}\volume{0000}

\maketitle

\label{firstpage}

%%%%%%%%%%%%%%%%%%%%%%%%%%%%%%%%%%%%%%%%%%%%%%%%%%%%%%
\begin{abstract}

  We present a detailed comparison of the substructure properties of a
  single Milky Way sized dark matter halo extracted by a variety of
  different (sub-)halo finders for simulations of cosmic structure
  formation. These finders span a wide range of techniques and employ
  different methodologies to identify and quantify substructures
  within a larger non-homogeneous background density (e.g. a host
  halo). We include both real-space and phase-space based finders as
  well as finders employing a Voronoi tessellation, the
  friends-of-friends technique or refined meshes as the starting point
  for locating substructure. The dataset, a part of the Aquarius suite
  (Springel et al.), simulates the same Milky Way type dark matter
  halo across a range of 5 different resolutions. This also allows us
  to perform convergence studies of the subhalo properties of the
  different finders.

  We applied a common post-processing pipeline to uniformly analyse
  the list of particles associated with each substructure as defined
  by the individual halo finders.  We extract quantitative and
  comparable measures for the \subhalos. We find that all of the
  finders agree extremely well on the presence and location of
  substructure. For properties relating to the inner part of the
  subhalo such as the maximum value of the rotation curve (and the
  actual location as also determined by our specific pipeline), the
  agreement between different finders is amazingly good. For
  properties that rely on particles near the outer edge of the subhalo
  (here defined as either $200$ or $500\times\rho_{\rm crit}$ with
  each subhalo treated in isolation) the agreement is at around the 20
  per cent level. We find that basic properties (mass, maximum
  circular velocity) of a subhalo can be reliably recovered if the
  subhalo contains more than 100 particles although its presence can
  be reliably inferred down our choice for a lower particle number
  limit of 20.

  The most difficult region within which to identify substructure is
  expected to be very close to the centre of the host halo where the
  background density is very high. Contrary to expectations none of
  the phase spaced based finders proved to be significantly better
  than the more traditional real-space based finders at recovering
  substructure in this region. In practice this volume contributes
  only a very small fraction to the typical total region of interest.

  We finally note that the logarithmic slope of the subhalo cumulative
  number count is remarkably consistent and $<1$ for all the finders
  that reached high resolution. This indicates that the larger and
  more massive, respectively, substructures are the most dynamically
  interesting and that higher levels of the (sub-)subhalo hierarchy
  become progressively less important.
\end{abstract}
%%%%%%%%%%%%%%%%%%%%%%%%%%%%%%%%%%%%%%%%%%%%%%%%%%%%%%

\begin{keywords}
methods: $N$-body simulations -- 
galaxies: haloes -- 
galaxies: evolution -- 
cosmology: theory -- dark matter
\end{keywords}

%%%%%%%%%%%%%%%%%%%%%%%%%%%%%%%%%%%%%%%%%%%%%%%%%%%%%%
\section{Introduction} \label{sec:introduction}

The growth of structure via a hierarchical series of mergers is now a
well established paradigm \citep{white_1978}. As larger structures
grow they subsume small infalling objects. However the memory of the
existence of these substructures is not immediately erased, either in
the observable Universe (where thousands of individual galaxies within
a galaxy cluster are obvious markers of this pre-existing structure)
or within numerical models, first noted for the latter by
\citet{klypin_overmerging_1999}.

Knowing the properties of substructure created in cosmological
$N$-body simulations allows the most direct comparison between these
simulations and observations of the Universe. The fraction of material
that remains undispersed and so survives as separate structures within
larger halos is an important quantity for both studies of dark-matter
detection \citep{zavala_2010} and the apparent overabundance of
substructure within numerical models when compared to observations
\citep{klypin_1999,moore_1999}. The mass and radial position of the
most massive Milky Way satellites seem to raise new concerns for our
standard $\Lambda$CDM cosmology
\citep{boylan_toobig_2011,boylan_mwlcdm_2011,dicintio_2011,ferrero_2011},
while differences between the simulated and observed internal density
profiles of the satellites seems to have been reconciled by taking
baryonic effects into account \citep[e.g.][]{Oh_2011,pontzen_2011}.
We are certain that between 5 per cent and 10 per cent of the material
within simulated galactic sized haloes exists within bound
substructures \citep[e.g.][]{contini_2011} and a substantial part of
the host halo has formed from disrupted subhalo material
\citep[e.g.][]{gill_2004b,knebe_2005,warnick_2008,cooper_2010,libeskind_2011}.

%not quite sure this is the right structure for the introduction and certainly other topics /citations can be added. Unfortunately I did not get any further with this. HL

Quantification of the amount of substructure (both observationally and
in simulations of structure formation) are therefore an essential tool
to what is nowadays referred to as ``Near-Field Cosmology''
\citep{freeman_2002} and attempts to do so in numerical models have
followed two broad approaches: either a small number of individual
haloes are simulated at exquisite resolution
\citep[e.g.][]{diemand_vl2_2008, springel_aquarius_2008,
  stadel_ghalo_2009} (respectively the via Lactea, Aquarius and GHalo
projects) or a larger representative sample of the Universe is
modelled in order to quantify halo-to-halo substructure variations
\citep[e.g.][]{angulo_2009, boylan_mill2_2009, klypin_bolshoi_2011}
who used the Millennium simulation, Millennium II simulation and the
Bolshoi simulation, respectively. As in this paper we have studied the
convergence of halo-finders within a single halo we can add nothing to
the topic of halo-to-halo substructure variations. In a very
comprehensive study that included 6 different haloes and 5 levels of
resolution \citet{springel_aquarius_2008} utilised their substructure
finder \subfind\ to detect around 300000 substructures within the
virial radius of their best resolved halo. They found that the number
counts of substructures per logarithmic decade in mass falls with a
power law index of at most 0.93, indicating that smaller substructures
are progressively less dynamically important and that the central
regions of the host dark matter halo are likely to be dominated by a
diffuse dark matter component composed of hundreds of thousands of
streams of tidally stripped material. \citet{maciejewski_2011}
confirmed the existence and properties of this stripped material using
a 6-dimensional phase space finder \hsf. A similar power law index was
also found for the larger cosmological studies of
\citet{boylan_mill2_2009, angulo_2009}. For the Bolshoi simulation
\citet{klypin_bolshoi_2011} find results that are in agreement with
their re-analysis of the Via Lactea II of \citet{diemand_vl2_2008}
with a abundance of \subhalos\ falling as the cube of the subhalo
rotation velocity. Rather than the present value of the maximum
rotation velocity they prefer to use the value that the subhalo had
when it first became a subhalo (i.e. on infall). This negates the
effects of tidal stripping and harassment within the cluster
environment but makes it difficult for us to directly compare because
we have generally only used the final $z=0$ snapshot for this
comparison study.

In recent years there has not only been a plethora of different groups
performing billion particle single-halo calculations, there has also
been an explosion in the number of methods available for quantifying
the size and location of the structures within such an $N$-body
simulation. In this paper we extend the halo finder comparison study
of \citet{knebe_haloes_2011} to examine how well these finders extract
the properties of those haloes that survive the merging process and
live within larger haloes. While this issue has already been addressed
by \citet{knebe_haloes_2011} it was nevertheless only in an academic
way where controlled set-ups of individual \subhalos\ placed into
generic host haloes were studied; here we however apply the comparison
to a fully self-consistently formed dark matter halo extracted from a
cosmological simulation. As the results of credible and reliable
subhalo identification have such important implications across a wide
range of astrophysics it is essential to ask how well the subhalo
finders do at reliably extracting the \subhalos. This still leaves the
open question of how well different modern gravity solvers compare
when performing the same simulation but at least we can hope to
ascertain whether or not -- given the same set of simulation data --
the different finders will arrive at the same conclusions about the
enclosed subhalo properties. We intend this paper to form the first
part of a series of comparisons. It primarily focuses on the most
relevant subhalo properties, i.e. location, mass spectrum and the
distribution of the value of the peak of the \subhalos\' rotation
curve.

In \S\ref{sec:Finders} we begin by summarising the eleven substructure
finders that have participated in this study, focusing upon any
elements that are of particular relevance for substructure finding. In
\S\ref{sec:Data} we introduce the Aquarius dataset that the described
finders analysed for this study. Both a qualitative and a quantitative
comparison between the finders is contained in \S\ref{sec:Comparison}
which also contains a discussion of our results, before we summarise
and conclude in \S\ref{sec:Summary}.

\begin{figure*}
 \includegraphics[height=\figwidth]{plots/AdaptaHOP-density-4}
 \includegraphics[height=\figwidth]{plots/AHF-density-4} \\
 \includegraphics[height=\figwidth]{plots/H3D-density-4}
 \includegraphics[height=\figwidth]{plots/H6D-density-4} \\
 \includegraphics[height=\figwidth]{plots/HBT-density-4} 
 \includegraphics[width=\figwidth]{plots/HSF-density-4}
\caption{The images show the smoothed dark matter density within an
  octant at level 4. In each panel the overplotted circles indicate
  the location of the recovered \subhalos\ for the finder labelled at
  the top of each panel. They are scaled proportionally using
  \vmax. Only \subhalos\ with a \vmax\ greater than 10 km/s are
  shown.}
\label{fig:haloes}
\end{figure*}

\begin{figure*}
\ContinuedFloat
 \includegraphics[height=\figwidth]{plots/mendieta-density-4} 
 \includegraphics[height=\figwidth]{plots/Rockstar-density-4} \\
 \includegraphics[height=\figwidth]{plots/STF-density-4} 
 \includegraphics[height=\figwidth]{plots/subfind-density-4} \\
 \includegraphics[height=\figwidth]{plots/VOBOZ-density-4} 
\caption[]{(continued) Recovered subhalo locations and sizes by
  labelled finder.}
\label{fig:haloes2}
\end{figure*}

%%%%%%%%%%%%%%%%%%%%%%%%%%%%%%%%%%%%%%%%%%%%%%%%%%%%%%
\section{The Merger Trees Builders} \label{sec:Finders} In this section we
are going to present the merger trees builders participating in the
comparison project in alphabetical order. Please note that we
primarily only provide references to the actual code description
papers and not an exhaustive portrait of each builder as this would be
far beyond the scope of this paper. While the general mode of
operation can be found in elsewhere, we nevertheless focus here on the
way each code collects and defines the set of particles belonging to a
subhalo: as already mentioned before, those particle lists are
subjected to a unique post-processing pipeline and hence the retrieval
of this list is the only relevant piece of information as far as the
comparison in this particular paper is concerned.

%put participants' details here

\subsection{Consistent Trees}

The \textsc{Consistent Trees} algorithm \citep{BehrooziTree} simulates
the gravitational bulk motion of halos given their positions,
velocities, and mass profiles as returned by the halo finder.  From
halos in any given simulation snapshot, the expected positions and
velocities of halos at an earlier snapshot may be calculated.  In some
cases, obvious inconsistencies arise between the predicted and actual
halo properties, such as missed satellite halos (e.g., satellite halos
which pass too close to the center of a larger halo to be detected)
and spurious mass changes (e.g., satellite halos which suddenly
increase in mass due to temporary misassignment of particles from the
central halo).  These defects can be simply repaired by substituting
predicted halo properties instead of the properties returned by the
halo finder.  This process helps to ensure accurate mass accretion
histories and merger rates for satellite and central halos; full
details of the algorithm as well as tests of the approach may be found
in \cite{BehrooziTree}.

\subsection{HBT}

HBT (\cite{han_etal_2012}) is a tree-oriented subhalo finder which builds the merger tree as it identifies subhaloes through tracking. Specifically, starting from a set of isolated haloes,  meger trees up to halo level are first constructed. HBT then traverses the halo merger trees from the earliest to the latest time and identifies a self-bound remnant for every halo at every snapshot after infall. These self-bound remnants are defined as descendent subhaloes of their progenitors. With this kind of tracking, each subhalo has at most one progenitor, which defines its main branch. The main branch extends until the subhalo goes under resolution. As soon as this happens, a final tracking step is done to determine a host subhalo into which it vanishes. After this the subhalo joins the branch of its host subhalo.

The major challenge in this method is to robustly track the subhaloes over long periods, and HBT has been specially tuned to meet this.  In addition, the merging hierarchy among progenitor haloes are utilized to efficiently allow satellite-satellite mergers or satellite accretion among subhaloes.

In order to apply HBT to existing subhalo catalogues, we first blind ourselves to the subhaloes and run HBT normally with only the haloes as input. When this is done, we match the HBT subhaloes to the provided ones. This enables us to map the HBT merger trees onto the provided subhalo catalogues.

\subsection{JMerger}
This algorithm runs by using only the gross data available for each halo. It compares two snapshots at a time, together with the time interval to account for the velocity calculation. For the two sets of haloes at t1 and t2, a new position is calculated for the centre of each halo by moving the t1 haloes forward in time by half the timestep, and the t2 haloes backwards by half the timestep.

Then for each halo in t1, a best match on position is found, together with constraints on mass and Vmax parameters. Mass is allowed to shrink by a given factor, and to grow by another factor.  Empirical values of 0.7 shrinkage and 4 growth are allowed. 
The Vmax parameter can only vary by +/- 70\%. The distance search is limited to twice the virial radius + four times the distance the halo has moved during the timestep.
This process attempts to trace haloes growing over time.

For those haloes that don't find a successor at t2, two other processes are tried. 
Firstly a major merger event, whereby two haloes are attempted to merge, and still fall within the distance and mass constraints of the first step.

Secondly minor mergers are considered whereby unmatched haloes are attempted to be merged with those haloes already matched in a minor merger process again using position and mass matching data.

\subsection{\textsc{MergerTree}}
\textsc{MergerTree} (publicly available with the \textsc{AHF} halo finder package, \texttt{http://popia.ft.uam.es/AHF}) is a very simple and straight forward particle correlator (despite its maybe misguiding name): it only takes two particle ID lists (preferentially coming from an \textsc{AHF} analysis) and identifies for each object in list \#1 those objects in list \#2 with which there are particles in common. This information is written into file. The code also writes a second file which assigns a unique "progenitor" to each object in file #1 as found in file \#2. To this extent a maximization of the merit function $M_i = N^{shared}_{i}^2/(N*N_i)$\footnote{$N^{shared}_{i}$ is the number of shared particles between the halo and its $i$-th progenitor, $N$ is the number of particles in the halo, and $N_i$ is the number of particles in the $i$-th progenitor} has proven extremely successful (Klimentowski et al. 2010; Libeskind et al. 2010; Knebe et al. 2013). The code can hence not only be used to trace a particular object backwards (or forward, depending on the temporal ordering of file \#1 and \#2) in time, but also to cross-correlate different simulations (e.g. different cosmological models run with the same phases for the initial conditions, etc.). But to be applicable to semi-analytical modeling it is desirable to ensure that each object not only has a unique progenitor but also a unique descendant. This is guaranteed by running \textsc{MergerTree} in a novel mode that applies the same merit function not only in one way but in both ways when correlating two files. For the comparison presented here \textsc{MergerTree} has been run in both modes, once without ensuring unique descendants and once with this feature activated. It further needs to be noted that the choice of the applied merit function eliminates the need for particles to only belong to one object: $M_i$ takes care of particles being assigned to multiple objects and was designed to do so in the first place, respectively.


\subsubsection{TreeMaker}

The code {\tt TreeMaker} was first used for semi-analytical model {\tt GalICS}  (Galaxies
in Cosmological Simulations) \citep{Hatton2003}. It was
first used on Frends-of-Frends halos\citep{Davis1985}, and later
applied on haloes and subhaloes extracted with group finder {\tt
  AdaptaHOP}\citep{Aubert2004,Tweed2009}.
The code simply consists in associating groups from two consecutive
time steps,  listing all progenitors (including the background) and
descendants (multiple descendants being allowed, the background being
ignored).
This first step is completed by using the particle ids within each
groups as working tracers. We mention that a particle can trace one single
group at a given step, meaning a particle in a subhalo would trace
only the subhalo but not the host halo. We thus refer to this
subset of particles -- unique to on individual group at one time step -- as tracers.

In order to create a ``usable'' merger tree a simplification stage is
required. Only one descendant per group (either halo
or subhalo) is selected and the list of progenitors updated to reflect
this selection. Selecting this unique descendent requires the use of a
merit function. The first versions of  {\tt TreeMaker} used a shared
merit function. For this study, we tested two modifications of
this selection, we added a normalized merit function $N_{\rm                                                                                                                                                       
  ij}/(N_{\rm i}N_{\rm j})$ where $N_{\rm ij}$ is the number of
particles found in both groups i and j in two consecutive time steps,
$N_{\rm i}$ and $N_{\rm j}$ being the respective number of tracers in these
groups. Since the data made available for this comparison
project contains the total energy of each particle within a group, we tested a possibility to weight each tracer of a given
group accordingly to its rank order in term of decreasing total energy. More
precisely for a group of $N$ tracers, the most bound tracer has a weigh
of $w_1=2N/(N+1)$ and the least bound a weight of
$w_N=2/(N+1)$. The total $\sum_{\rm i}w_{\rm i}$ being $N$ by definition, the shared merit function is then written as
$\sum_{\rm i=j} w_{\rm i}$ and the normalized shared merit function as
$\sum_{\rm i=j} w_{\rm i}/(N_{\rm i}N_{\rm j})$. The progenitors
themselves are also ordered by decreasing merit function, so that the
main progenitor itself has the highest value of the given merit function.

We different runs used in this paper are labeled as follows:
\begin{enumerate}
  \item {\tt TreeMaker\_v2}: Shared merit function, no tracer weighting
  \item {\tt TreeMaker\_v3}: Normalized shared merit function, no
    tracer weighting
  \item {\tt TreeMaker\_v4}: Shared merit function, tracer weighting
  \item {\tt TreeMaker\_v5}: Normalized shared merit function, tracer weighting
\end{enumerate}




\begin{figure*}
  %\includegraphics[width=\resplot]{plots/AHF-density-4-nolab} 
 %\includegraphics[width=\resplot]{plots/Rockstar-density-4-nolab}
  %\includegraphics[width=\resplot]{plots/subfind-density-4-nolab}\\
  \includegraphics[width=\resplot]{plots/AHF-density-3-nolab} 
\includegraphics[width=\resplot]{plots/Rockstar-density-3-nolab}
  \includegraphics[width=\resplot]{plots/subfind-density-3-nolab}\\
  \includegraphics[width=\resplot]{plots/AHF-density-2-nolab} 
\includegraphics[width=\resplot]{plots/Rockstar-density-2-nolab}
  \includegraphics[width=\resplot]{plots/subfind-density-2-nolab}\\
  \includegraphics[width=\resplot]{plots/AHF-density-1-nolab} 
\includegraphics[width=\resplot]{plots/Rockstar-density-1-nolab}  
  \includegraphics[width=\resplot]{plots/subfind-density-1-nolab}\\
\caption{Subhalo recovery as a function of resolution. Location and
  size of recovered substructure from level 3 to level 1 for the three
  finders that reached this level. In all panels \subhalos\ with
  \vmax $>10$ km/s are shown, scaled by \vmax\ as in \Fig{fig:haloes} and
  the background image is the smoothed dark matter density at that
  level. The relevant finder and level are labelled in the top right
  of each panel. The biggest change between levels is the additional
  small scale power moving the substructure locations.}
\label{fig:plotcomp}
\end{figure*}

%%%%%%%%%%%%%%%%%%%%%%%%%%%%%%%%%%%%%%%%%%%%%%%%%%%%%%
\section{The Data} \label{sec:Data}

The data used for this paper are from (Milgas mini-run) 

The data used for this paper forms part of the Aquarius project
\citep{springel_aquarius_2008}. It consists of multiple dark matter
only re-simulations of a Milky Way like halo at a variety of
resolutions performed using \gadget\
\citep{springel_cosmological_2005}.  We have used the Aquarius-A halo
dataset for this project.  This provided 5 levels of data, varying in
complexity from the 2.3 million particles of the lowest resolution
level 5 up to the 4.25 billion particles of the highest resolution
level 1 as shown in Table~\ref{tbl:Data}. The underlying cosmology for
the Aquarius simulations is the same as that used for the Millennium
simulation \citep{springel_millenium_2005} i.e. $\Omega_M = 0.25,
\Omega_\Lambda = 0.75, \sigma_8 = 0.9, n_s = 1, h = 0.73$. These
parameters are consistent with the latest WMAP data \citep{wmap_2011}
although $\sigma_8$ is a little high.  All the simulations were
started at an initial redshift of 127. Precise details on the set-up
and performance of these models can be found in
\citet{springel_aquarius_2008}.

The participants were asked to run their subhalo finders on the
supplied data and to return a catalogue listing the substructures they
found.  Specifically they were asked to return a list of uniquely
identified substructures together with a list of all particles
associated with each subhalo.

Finders were initially run on the smallest dataset, the Aq-A-5 data.
This allowed for debugging of the common output format required by the
project and some basic checks on the internal consistency of the data
returned from each participant. Once this had been achieved each
participant scaled up to the higher resolution datasets, continuing
until they reached the limits of their finder and/or the computing
resources readily available to them. A summary of the number of
\subhalos\ found by each subhalo finder at the various levels is
contained in Table~\ref{tbl:ResultsCent} as well as the size of the
largest subhalo at level 4. All of the finders that participated in
this study completed the analysis of the level 4 dataset which is used
for the main comparison that follows and contains around 6 million
particles within the region considered, a sphere of radius 250 kpc/$h$
around a fiducial centre. Three of the finders (\ahf, \rockstar\ \&
\subfind) completed the analysis of the very computationally demanding
level 1 dataset. In addition to these \hbt\ and \hsf\ completed level
2 which contains around 160 million particles within the region
examined here.

\begin{table*}
  \caption{Summary of key numbers for each Aquarius level, the dataset
    used for this study. $N_h$ is the number of particles with the
    highest resolution (lowest individual mass).  $N_l$ the number of
    low resolution particles - the sum of the remainder.  $N_{250}$ is
    the number of high resolution particles found within a sphere of
    radius 250 kpc/h from the fiducial centre at each resolution ({\it
    i.e.} those of interest for this study). $M_p$ is the mass of one
    of these particles (in $\mathrm{M_{\sun}/h}$). $S$ is the resolution increase
    (mass decrease) for each level relative to level 5, and $S_p$ is
    the resolution increase relative to the previous level.  All
    particles are dark matter particles.}

  \label{tbl:Data}
  \begin{tabular}{l r r r r r r }
    \hline
    Data & $N_h$ & $N_l$ & $N_{250}$ & $M_p$ & $S$ & $S_p$\\
    \hline
    Aq-A-5 &  2,316,893    & 634,793     & 712,232     & $2.294\times10^{6}$& 1    &$\times 1$\\
    Aq-A-4 & 18,535,972    & 634,793     & 5,715,467   & $2.868\times10^{5}$& 8    &$\times 8$\\
    Aq-A-3 & 148,285,000   & 20,035,279  &45,150,166   & $3.585\times10^{4}$& 64   &$\times 8$ \\
    Aq-A-2 & 531,570,000   & 75,296,170  &162,527,280  & $1.000\times10^{4}$& 229  &$\times 3.6$\\
    Aq-A-1 & 4,252,607,000 & 144,979,154 &1,306,256,871& $1.250\times10^{3}$& 1835 &$\times 8$\\
    \hline
  \end{tabular}
\end{table*}

\begin{table*}
  \caption{The number of \subhalos\ containing 20 or more particles and
    centres within a sphere of radius 250kpc/h from the fiducial
    centre found by each finder after standardised post-processing
    (see text). Three finders (\ahf, \rockstar\ \& \subfind) returned
    results from the highest resolution (level 1) within the timescale
    of this project. Below this we list the number of particles
    contained within the largest subhalo after post-processing.}
  \label{tbl:ResultsCent}
  \begin{tabular}{l *{11}{r}}
 \hline
         & \multicolumn{11}{c}{Number of \subhalos\ within 250kpc/h of
         the fiducial centre after post processing.}\\
    Name & \adaptahop&\ahf   &\htd&\hsd &\hbt &\hsf &\mendieta&\rockstar &\stf &\subfind&\voboz \\
    \hline 
    Aq-A-5 & 354     & 230   & 58 & 136 & 228 & 231 & 180     & 272      & 205 & 214    & 257 \\ 
    Aq-A-4 & 2497    & 1599  &1265&1075 & 1544& 1544& 985     & 1707     & 1520& 1433   & 1862 \\ 
    Aq-A-3 & -       & 11213 & -  & -   &11693&11240& 8048    & 11797    &10250& 10094  & 13343 \\ 
    Aq-A-2 & -       & 38441 & -  & -   &39703&20584& -       & 38489    & -   & 33135  & - \\ 
    Aq-A-1 & -       & 226802& -  & -   & -   & -   & -       & 235819   & -   & 221229 & - \\ 
    \hline
         & \multicolumn{11}{c}{Number of particles contained in
         largest \subhalos\ within 250kpc/h of
         the fiducial centre after post processing.}\\
    Aq-A-4 & 49076    & 77225 & 66470 & 69307 & 61581 & 73167 & 48320 & 78565 & 51874 & 50114 & 54685 \\ 
    \hline 
  \end{tabular}
\end{table*}

Both the halo finder catalogues (alongside the particle ID lists) and
our post-processing software (to be detailed below) are publically
available from the web site
\href{http://popia.ft.uam.es/SubhaloesGoingNotts}{http://popia.ft.uam.es/SubhaloesGoingNotts}
under the Tab ``Data''.


%%%%%%%%%%%%%%%%%%%%%%%%%%%%%%%%%%%%%%%%%%%%%%%%%%%%%%
\section{The Comparison} \label{sec:Comparison}

We are going to primarily focus on comparing the location of \subhalos\
(both visually and quantitatively), the mass spectrum, and the
distribution of the peak value of the rotation curve. The comparison,
however, is based solely upon the provided particle lists and not the
subhalo catalogues as the latter are based upon each code's own
definitions and means to determine aforementioned properties and hence
possibly introducing ``noise'' into the comparison
\citep[cf.][]{knebe_haloes_2011}. In order to achieve a fair
comparison between the respective finders we produced a single
analysis pipeline which we used to post-process the particle lists
provided by each participating group. This ensured consistency across
our sample while removing differences due to the adoption of different
post-processing methodologies and the particular choice of threshold
criteria. The comparison detailed in this paper is restricted to this
uniform post-processed dataset. We intend to explore differences due
to different methodologies in a subsequent work. However, we stress at
the outset that our particular chosen post-processing methodology is
not intended to be unique nor do we put it forward as the {\it best}
way of defining a subhalo. Rather we use a single methodology so that
we can first answer the most fundamental question: if we agree on a
single subhalo definition do the different finders agree on the most
fundamental properties they recover?  Perhaps surprisingly we will see
that the answer to this question is broadly yes.

\subsection{Post-processing pipeline}

Some finders (e.g. \ahf) include the mass (and particles) of a subhalo
within the associated host halo whereas others do not (e.g. \subfind),
preferring each particle to only be associated with a single halo or
subhalo. Either of these approaches has its pros and cons. For
instance, keeping the subhalo mass as part of the halo mass makes it
straightforward to calculate the enclosed dynamical mass of any
object. However, such an approach easily leads to multiple counting of
mass, particularly if there are many layers of the substructure
hierarchy. Halo-finding authors rightly argue that it is not difficult
to transform from one definition to the other given that you in
principle know both the halo and particle locations. In our study 5 of
the 11 finders chose to include the mass of \subhalos\ whereas the
other 6 did not. Following our principle of creating a uniform
analysis pipeline we processed all the particle lists to insist that a
particle could only reside within a single subhalo and hence removing
sub-\subhalos\ from \subhalos, etc.  To this extent, we first sorted
the returned halo catalogue into mass order. Then starting from the
smallest halo we performed the centring, trimming and overdensity
checks detailed below to trim the subhalo uniformly. We then tagged
the particles contained within this object as being within a subhalo
before continuing to the next largest subhalo and repeating the
procedure ignoring particles already tagged as being used before. This
preserved the maximum depth of the subhalo hierarchy while ensuring
that a particle could only reside within a single subhalo. We though
have to remark that in practice excising all the sub-\subhalos\ from
each subhalo's particle list made little difference to any of the
results presented here as at any level of the subhalo hierarchy only
around 10 per cent of the material is within a subhalo of the current
halo. So sub-\subhalos\ contribute only around 1 per cent of the halo
mass.

All the particles belonging to the list each finder identified as
being associated with a subhalo were extracted from the original
simulation data files to retrieve each particle's position, velocity
and mass. From this data the centre of mass was first calculated,
before being refined based on considering only the innermost 10 per
cent of these particles sorted with respect to the initial centre of
mass. This procedure was repeated until a stable centre was found,
i.e. until the change in the position was below the actual force
resolution of the simulation. Once the centre had been defined the
particles were ordered radially from this point and a rotation curve
$GM(<r)/r$ and overdensity $M(<r)/(4\pi r^3/3)$ calculated until it
dropped below 200 times the background density (\rth) defining the
subhalo radius $R_{200}$ and mass $M_{200}$. All particles outside
$R_{200}$ were removed which was essential in particular for the
phase-space finders who also considered already stripped material as
still being part of and belonging to the subhalo. At this point the
maximum circular velocity, \vmax\ was obtained by smoothing the
rotation curve and locating its maximum by searching both inwards and
outwards for a peak in the rotation curve and taking the average of
these two measures, a process that stabilises the measure if the
rotation curve is very flat. 

We emphasise that the precise subhalo properties are somewhat
sensitive to the definition of the halo centre. Various groups use the
centre-of-mass as the centre of all material enclosed within the
subhalo's radius (both with and without including substructure), the
centre-of-mass of some smaller subset (as here for example), the
location of the most bound particle or the location of the densest
particle. Additionally different groups use different methodologies
for deciding whether or not a particle is bound to a halo as this
involves some decisions about the global potential and can be a very
time consuming process if done fully generally and
iteratively.

Finally the choice of where to place the subhalo edge is also
problematic. By definition the subhalo resides within some
in-homogeneous background density and so at some point particles cease
to belong to it and should rather be associated with the background
object. Different groups split the host halo from the subhalo in
different ways and there is no {\it correct} method. Without a uniform
choice these differences can swamp any differences due to actually
finding \subhalos\ or not. We stress that our post-processing (where
we treat each subhalo in isolation) can only remove particles from the
original list of those particles associated with a subhalo. We have
therefore tested whether or not our results are sensitive to our
choice of 200 as an overdensity parameter by re-running our analysis
with a tighter threshold of 500. Other than making all the subhalo
masses smaller this has no noticeable effect on the scatter of the
cumulative number counts. We therefore decided to stick to the
original choice of $R_{200}$ and $M_{200}$, respectively. Further,
throughout the subsequent comparison only haloes with more than 20
(bound) particles within \rth were used, although some finders
detected and returned haloes with less particles.

To summarise, our uniform post-processing pipeline involved the
following steps, applied iteratively where necessary:
\begin{itemize}
\item The subhalo catalogues were sorted into mass order.
\item Starting from the smallest subhalo, the particles associated with the
  current subhalo were obtained from the simulation data.
\item Only particles tagged as ``not used before'' were considered.
\item The centre-of-mass was iteratively calculated using the
  innermost 10\% of particles.
\item A value for \rth\ was calculated based on an enclosed overdensity of 200
  times the cosmic background density. 
\item The subhalo mass and rotation curve peak \vmax\ were computed based
  on particles inside \rth.
\item Only substructures containing more than 20 particles were retained.
\end{itemize}

\subsection{Visual comparison}

A visual representation of the location and size of the recovered
\subhalos\ at Aquarius level 4 from each of the finders is shown in
\Fig{fig:haloes}. A smoothed colour image of the underlying dark
matter density in one octant of the main halo is overplotted and
\subhalos\ are characterised by circles whose size is scaled according
to \vmax\ (specifically \vmax\ (in km/s) divided by 3). This allows a
visual comparison between the finders. Only haloes with \vmax\ $>
10h^{-1}$km/s are indicated. We immediately see that most of the
finders are very capable, ably extracting the locations of the obvious
overdensities in the underlying dark matter field. Wherever you would
expect to find a subhalo (given the background density map) one is
indeed recovered. This demonstrates that substructure finders should
be expected to work well, recovering the vast majority of the
substructure visible to the eye. Additionally, if a uniform
post-processing pipeline is applied the quantitative agreement between
the finders is also excellent, with the extracted structures having
very similar properties between finders (see below). The majority of
the finders agree very well, reliably and consistently recovering
nearly all the \subhalos\ with maximum circular velocities above our
threshold.

\Fig{fig:haloes} illustrates the agreement between the finders at a
single Aquarius level (in this case level 4, which all the
participating finders have completed). In \Fig{fig:plotcomp} we
construct a similar figure to illustrate the agreement between
levels. We show the same quadrant at level 3 to level 1 for the three
finders (\ahf, \rockstar\ \& \subfind) that have completed the level 1
analysis; we deliberately omitted level 5 and level 4 as the former is
not very informative and the latter has already been presented in
\Fig{fig:haloes}. As can be seen, the main difference between the
different levels is in the exact location of the substructures. This
changes because additional power was added to the Aquarius initial
power spectrum to produce the additional small objects that form as
the resolution is increased (fundamentally, the Nyquist frequency has
changed as there are more available tracers within the higher
resolution box). This extra power moves the substructure around
slightly, and these differences are amplified in the, by definition,
non-linear region of a collapsed object. Despite this the ready
agreement between the three finders at any single level is clear to
see and this is similarly true for both the other finders (\hbt, \hsf)
that completed level 2. We do not explore the effect of changing the
resolution on subhalo extraction in more detail here because that is
not the main point of this paper, which focuses on how well different
finders extract substructure relative to each other. Also, this topic
has already been well studied for \subfind\ using this same suite of
models by \citep{springel_aquarius_2008}.

\subsection{Subhalo Mass Function}

\begin{figure*}
 \centering
 \includegraphics[width=1\linewidth]{plots/submassfunc} \\
\caption{ Cumulative number count of \subhalos\ above the indicated mass
found within a radius of 250 kpc/h from the fiducial halo centre after
standardised post-processing at resolution level 4 (see text for details).}
\label{fig:massfunc}
\end{figure*}

\subsubsection{Level 4}
Perhaps the most straightforward quantitative comparison is simply to
count the number of \subhalos\ found above any given mass. For
Aquarius level 4 this produces the cumulative mass plot shown in
\Fig{fig:massfunc}. Results from each participating finder are shown
as a line of the indicated colour. Generally the agreement is good,
with some intrinsic scatter and a couple of outliers (particularly
\adaptahop\ and \mendieta ) which do not appear to be working as well
as the others, finding systematically too many or too few \subhalos\
of any given mass respectively. Typically the scatter is around the 10
per cent level except at the high mass end where it is larger as each
finder systematically recovers larger or smaller masses in general. We
like to remind the reader again that this scatter is not due to the
inclusion/exclusion of sub-\subhalos\ (which has been taken care of by
our post-processing pipeline) nor the definition of the halo edge: the
10\% differences remain if choosing $R_{500}$ as the subhalo edge.

Table~\ref{tbl:ResultsCent} lists the number of \subhalos\ found that
contain 20 or more particles after the uniform post-processing
procedure detailed above had been performed and within $250$ kpc/h of
the fiducial centre\footnote{We adopted a fixed and unique position
  for the host halo of $x=57060.4,y=52618.6,z=48704.8$ kpc/$h$.} of
the main Aquarius halo at each level completed for all the eleven
finders that participated.  These number counts are generally
remarkably consistent, again with a few outliers. The majority of the
finders are recovering the substructure locations well.

As an additional quantitative comparison we list the number of
particles associated with the largest substructure found by each of
the finders as the last row of Table~\ref{tbl:ResultsCent}. All the finders
recover a structure containing $60,000$ particles $\pm 20$ per cent. As
\Fig{fig:massfunc} has shown above there is a lot of residual scatter
for the highest mass haloes even when a uniform post-processing
pipeline is used. This is most likely due to the different unbinding
algorithms used in the initial creation of the substructure membership
lists which are particularly uncertain for these large structures. At
the other end of the substructure mass scale we have chosen to
truncate our comparison at \subhalos\ containing 20 particles as this
was shown to be the practical limit in \cite{knebe_haloes_2011}. Some
participants returned halos smaller than this as this is their normal
practice. They all stress that such small \subhalos\ should be treated
with extreme caution but that there does appear to be a bound object
at these locations even if its size is uncertain. We have removed them
here for the purposes of a fair comparison.

\begin{figure}
 \centering
  \includegraphics[width=1\linewidth]{plots/submassfuncM-convergence}
  \caption{Cumulative subhalo mass function (multiplied by $M$ to
    compress the vertical dynamical range) for all five Aquarius
    levels for the \ahf, \rockstar, and \subfind\ finder. We fit the
    function $N/N_{tot} = a_0 \times M^{-n}$ between the mass
    equivalent to 100 particles at each level and $10^9
    \mathrm{M_{\sun}/h}$.}
 \label{fig:fitAHF}
\end{figure}

\begin{figure}
 \centering
  \includegraphics[width=1\linewidth]{plots/fitsubmass}
  \caption{A comparison of the slope and normalisation of the fits
    derived as per \Fig{fig:fitAHF} for all finders at all levels
    returned.}
 \label{fig:fitsubmass}
\end{figure}

\subsubsection{All Levels}
Cumulative subhalo number counts like that shown for level 4 in
\Fig{fig:massfunc} can be calculated for all levels completed and
compared. As shown in \Fig{fig:plotcomp} while increasing the
resolution does not exactly reproduce the same substructures a
reasonable approximation is achieved and so we expect to find a set of
similar \subhalos\ containing more particles as we decrease the
individual particle mass between levels (i.e. any specific subhalo
should effectively be better resolved as the resolution increases). We
show the cumulative number counts for the finders \ahf, \rockstar, and
\subfind\ (multiplied by $M$ to compensate for the large vertical
scale) from level 5 to level 1 in \Fig{fig:fitAHF}. We show this as an
example and stress that similar plots with equally good agreement and
similar features could be produced for any of the finders that
completed level 2. The curve for each level starts at 20 particles per
halo. Below about 100 particles per halo the cumulative number counts
fall below the better resolved curves, indicating that \subhalos\
containing between 20 and 100 particles are not fully resolved and
should have a slightly higher associated mass, also reported in
\citet{muldrew_accuracy_2011}.  Above $10^9 \mathrm{M_{\sun}/h}$ the
power law slope breaks as there are less than 10 \subhalos\ more
massive than this limit and the number of these is a property of this
particular host halo. For these reasons we fit a power law of the form
\begin{equation}\label{eq:massfit}
  \frac{N}{N_{tot}} = a_0 M^{-n}
\end{equation}
between 100 particles and $10^9 \mathrm{M_{\sun}/h}$ where the power
law breaks.  Here $a_0$ is a normalisation, $M$ is the mass and $n$ is
the power law slope. The fitted values of the parameters by level are
given in the legend for each finder. The subhalo cumulative number
count appears to be an unbroken power law -- at least in the range
considered for the fitting. Similar results for \subfind\ were found
by \citet{springel_aquarius_2008}. 

We extended this particular analysis of fitting a single power-law to
the (cumulative) subhalo mass function to all finders at all available
levels and compare the values of $a_0$ and $n$ as a function of level
for all participating substructure finders. \Fig{fig:fitsubmass} shows
that at level 5 little can be said because the fitting range is so
narrow. At the lower levels good agreement is seen between the finders
and a consistent trend emerges: all agree that the power law slope,
$n$ is less than 1 and if anything decreasing with increasing
resolution. Clearly both \adaptahop\ and \htd\ are strong outliers on
this plot. Values of $n$ less than 1 are significant because they
imply that not all the mass is contained within substructures, with
some material being part of the background halo. This has important
ramifications for studies requiring the fraction of material within
substructures such as the dark matter annihilation signal and lensing
work. Although this result is robust between all high-resolution
finders we remind the reader that this is for a single halo within a
single cosmological model. However, it does indicate that, as perhaps
expected, the most important contribution to substructure mass is from
the most massive objects and that progressively smaller structures
contribute less and less to the signal.


\subsection{Distribution of  \vmax}
If, instead of quantifying the total mass of each subhalo, we rather
use the maximum rotational velocity, \vmax\ to rank order the
\subhalos\ in size we obtain a generally much tighter
relation. \Fig{fig:vmaxplot} displays the cumulative \vmax\ for all
the finders for level 4 again. \citet{knebe_haloes_2011} already found
that \vmax\ was a particularly good metric for comparing haloes and we
confirm this for \subhalos. As \citet{muldrew_accuracy_2011} showed in
figure 6 of their paper, this is because for an NFW profile
\citep{nfw_1997} the maximum of the rotation curve is reached at less
than 20 per cent of the virial radius for objects in this mass range
so \vmax\ is a property that depends upon only the very inner part of
the subhalo and is not affected by any assumptions made about the
outer edge. Except for \stf\ all the finders align incredibly well for
the largest \subhalos\ with \vmax $>$ 20 km/s. For \subhalos\ smaller
than this the alignment remains tighter than the total mass comparison
down to rotation velocities of around 6 km/s. At level 4 haloes of
this size contain around 80 particles in total, so \vmax\ is being
calculated from less than 20 particles at this point. While it agrees
well with the other finders for high rotation velocities \adaptahop\
displays a strange behaviour for small values of \vmax, finding many
more small \subhalos\ than any of the other finders.

\begin{figure*}
 \centering
 \includegraphics[width=1\linewidth]{plots/vmaxfunc}
\caption{Cumulative number count of \subhalos\ above the indicated \vmax\
value within a radius of 250 kpc/h from the fiducial halo centre after
standardised post-processing (see text for details). The arrows
indicate the number of particles interior to $r_{\rm max}$, the
position of the peak of the rotation curve.}
\label{fig:vmaxplot}
 \end{figure*}

\subsection{Radial Mass Distribution}
The accumulated total mass of material within \subhalos\ is measured
by ordering the subhalo centres in radial distance from the fiducial
centre of the halo and summing outwards,i.e. $\sum_{r_{\rm sat}<r}
M_{\rm sat}$. We include all post-processed \subhalos\ above our mass
threshold of 20 particles. As \Fig{fig:radialdist} demonstrates at
level 4 most of the finders (\ahf, \hbt, \hsd, \hsf, \stf, \voboz)
agree very well, finding very similar amounts of substructure both in
radial location and mass. \rockstar\ finds a little more structure,
particularly in the central region where its phase-space nature works
to its advantage and \subfind\ finds around a factor of $25\$ less due
to its conservative subhalo mass assignment \textbf{AK: that needs to
be explained in the SUBFIND description in Section 2!}. Not
surprisingly \htd\ struggles in the very central region where it is
difficult for the underlying real-space friends-of-friends methodology
to distinguish structures from the background halo. The \mendieta\
finder appears to fail badly. As previously noted the \adaptahop\
finder locates many small \subhalos\ and these push up the total mass
found in substructure above that found by the others particular in the
range around 50-100kpc. We note that three of the four phase space
based finders (\hsd, \hsf\ & \stf) have a radial performance
indistinguishable from real-spaced based finders. The only one to show
any difference is \rockstar\ and it remains unclear whether or not
this is in practice a significant improvement.

\begin{figure*}
 \centering
 \includegraphics[width=1\linewidth]{plots/radialdist_M}
\caption{Cumulative plot of the enclosed mass within \subhalos\ as a
  function of radial distance from the fiducial centre of the host
  halo.}
\label{fig:radialdist}
\end{figure*}


%%%%%%%%%%%%%%%%%%%%%%%%%%%%%%%%%%%%%%%%%%%%%%%%%%%%%%
\section{Summary \& Conclusions} \label{sec:Summary}

We have used a suite of increasing resolution models of a single Milky
Way sized halo extracted from a self-consistent cosmological
simulation (i.e. the Aquarius suite \citep{springel_aquarius_2008}) to
study the accuracy of substructure recovery by a wide range of popular
substructure finders. Each participating group analysed independently
as many levels of the Aquarius dataset as they could manage and
returned lists of particles they associated with any subhalo they
found. These lists were post-processed by a single uniform analysis
pipeline. This pipeline employed a standard fixed definition of the
subhalo centre and subhalo mass, and employed a standard methodology
for removing particles and deriving \vmax. This analysis was used to
produce cumulative number counts of the \subhalos\ and examine how
well each finder was able to locate substructure.

We find a remarkable agreement between the finders which are based on
widely different algorithms and concepts. The finders agree very well
on the presence and location of \subhalos\ and quantities that depend
on this or the inner part of the halo are amazingly well and reliably
recovered. We agree with \citet{knebe_haloes_2011} that \vmax\ is a
good parameter by which to rank order the halos (in this case
\subhalos). However, we also show that as \vmax\ is only dependent
upon the inner 20 per cent or less of the subhalo particles around 100
particles are required to be within the subhalo for this measure to be
reliably recovered.  Quantities that depend on the outer parts of the
\subhalos, such as the total mass, are still recovered with a scatter
of around 10 per cent but are more dependent upon the exact algorithm
employed both for unbinding and to define the outer edge.

The most difficult region within which to resolve substructures is the
very centre of the halo which has, by definition, a very high
background density. In this region real-space based finders are
expected to struggle whereas the full six-dimensional phase space
based finders should do better. In practice \rockstar\ is the only
phase space based finder that shows any indication of this (and this
difference becomes {\it less} pronounced as the resolution is
increased).  We conclude that, as yet, non of the phase space based
finders present a significant improvement upon the best of the more
traditional real-space based finders.

Convergence studies indicate that identified \subhalos\ containing
less than 100 particles tend to be under-resolved and these objects
grow slightly in mass if a higher resolution study is used. This could
be due to that fact that particles in the outer regions of these
\subhalos\ are stripped more readily at lower resolution or it could
be an artifact of the difficulty of measuring the potential (and hence
completing any unbinding satisfactorily) with this small number of
particles. Several studies
\citep{kase_2007,pilipenko_2009,trenti_2010} have indicated the
unreliability of halo properties (other than physical presence) for
(sub)halos of this size or less.

Fitting power law slopes to the convergence studies of each finder
indicates that the logarithmic slope of the cumulative number count is
less than 1. While this is only for a single halo within a single
cosmology the result appears to be robust as it is found for all the
high-resolution finders employed in this study. This indicates that
the larger substructures are the most important ones and that higher
levels of the (sub)subhalo hierarchy play a less significant dynamical
role.


\section*{Acknowledgements} \label{sec:Acknowledgements}

We wish to thank the Virgo Consortium for allowing the use of 
the Aquarius dataset and Adrian Jenkins for assisting with the data.

AK is supported by the {\it Spanish Ministerio de Ciencia e
  Innovaci\'on} (MICINN) in Spain through the Ramon y Cajal programme
as well as the grants AYA 2009-13875-C03-02, AYA2009-12792-C03-03,
CSD2009-00064, and CAM S2009/ESP-1496. He further thanks Astrud
Gilberto for the shadow of your smile.

HL acknowledges a fellowship from the European Commission's Framework Programme 7, 
through the Marie Curie Initial Training Network CosmoComp (PITN-GA-2009-238356).

\bibliography{mn-jour,SubHaloes}
\bibliographystyle{mn2e} \label{sec:Bibliography}
\label{lastpage}


\end{document}
